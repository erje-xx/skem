\documentclass{article}
\usepackage{amssymb,amsmath}
\usepackage{graphicx}
\title{s[kf]em}

\begin{document}
\maketitle
\begin{abstract}
the s[kf]em setup.
\end{abstract}
\section{Overview}
A description of the magnetometer necessary for the interpretation of experimental results is presented.\cite{SKEM_Tamaru, SKEM_Back}
\section{Optical system}
The optical system consists of a second-harmonic generator, a beam expander and diaphragm, a beamsplitter, an oil-immersion objective lens, the sample, an optical retarder, a Wollaston prism, and two photodiodes.

\subsection{Laser}
A Ti:Sapphire laser, a Mira 900 from Coherent Inc., produces pulses of duration X, energy Y, central wavelength Z, and spectral width ZZ. The beam is polarized and modelled as monochromatic.

\subsection{Second-harmonic generator}
Halves wavelength, preserves polarization.

\subsection{Beam focusing}
Beam expander and diaphragm.

\subsection{Detector}
Wollaston prism, two balanced photodiodes, differential amplifier.

\subsection{Polarization analysis}
We aim to measure the change in the beam's polarization upon reflection from the sample. In our case, the angle of incidence is $\theta = 0$, therefore, the polar Kerr effect produces a change in polarization and ellipticity of the beam, proportional to the out-of-plane magnetization.

We can use the formalism of Jones' matrices to calculate the change of polarization and ellipticity upon reflection from the sample (see Appendix A). In the end, we arrive at a formula for the differential intensity detected at the photodiodes, our measured signal.

\begin{align}
\Delta I &= I_{x} - I_{y} \propto E_{(O)x}^* E_{(O)x} - E_{(O)y}^* E_{(O)y} \\
&= |r_{ss}|^2 \left[ \left(1 - \frac{|r_{ps}|^2}{|r_{ss}|^2}\right) \cos{2 \eta} - 2 \Re{\frac{r_{ps}}{r_{ss}}} e^{-i \delta} \sin{2 \eta} \right] \\
&\approx I_{0} [\cos{2 \eta} - 2(\theta_{K} \cos{\delta} - \epsilon_{K} \sin{\delta}) \sin{2 \eta}]
\end{align}
where the last approximation consists in considering small ellipsometric angles (cite lopusnik thesis!).

\subsection{Camera system?}

\subsection{Noise analysis}

\section{Excitation}
s[kf]em constitutes a modification of standard TR-MOKE techniques. In TR-MOKE, an impulse field excites the system and the magnetization response is measured by means of MOKE. Applying a Fourier transform to the recorded response enables one to map the frequency response of the system.

In S[KF]EM, the excitation field is sinusoidal. Synchronization between the laser pulse and excitation enables recording the magnetization response over an entire period of the excitation. 

A sinusoidal microwave field of frequency $f_{ex}$ sent through a loop or microstrip antenna generates the excitation field $h_{ex}$. An oven-controlled crystal oscillator generates a highly-stabilized 10 MHz reference signal, which acts as a reference to generate both the sinusoidal microwave field, as well as a 76 MHz signal used to phaselock the pulses to the CW microwave signal. To achieve phaselocking, the microwave frequency must be a multiple of both the laser repetition rate and the reference frequency, i.e. an integer multiple of 380 MHz. To improve the signal-to-noise ratio, we apply the microwave pulses for 150 $\mu$s at a repetition frequency of 3 kHz and a lock-in amplifier extracts this frequency component from the photoinduced current.

\subsection{Synchronization}
The principle of excitation/detection is based on stroboscopic imaging.

\subsection{Phase modulation}

\subsection{Lock-in detection}
Optical chopper, polarization modulation, electronic chopping.

\section{Positioning and Control}

\subsection{Stage}

\subsection{LabView/Stabilization}

%We detect the magnetization dynamics by stroboscopic imaging via the magneto-optical Faraday effect synchronized with the microwave field. 
%The approach used in the current study is very similar to that reported in Refs. 8 and 18. 
%An oven controlled crystal oscillator generates a highly stabilized 10 MHz reference signal, which acts as a reference to generate both a sinusoidal microwave field, used to excite the spin-wave modes, as well as a 76 MHz signal used to phaselock the pulses of the Ti:sapphire laser to the CW microwave signal. 
%The phase of the laser pulses is adjustable with respect to the 10 MHz reference signal generated by a tunable microwave synthesizer. 
%In order to achieve phase-locking of the microwave signal, the microwave frequency must be a multiple of both the laser repetition rate 76 MHz and the 10 MHz reference frequency, i.e an integer multiple of 380 MHz. 
%Throughout this paper we have used a microwave frequency of 3.8 GHz. 
%For synchronization of the lock-in detection scheme, a series of microwave pulses with duration of 150 μs and a repetition frequency of 3kHz have been applied to the excitation coil. 
%The average microwave power was varied from 10⁻⁵ mW to 1.6  x 10³ mW.
%
%The Ti-sapphire laser, a Mira 900 from Coherent Inc., has been operated at a wavelength of 820 nm, a pulse-duration of about 200 fs, pulse energy 0.013 nJ and a average power of less than 1 mW. 
%The s-polarized incident laser beam was first widened by a beam expander and then focused to the sample surface with an achromatic lens. 
%By scanning the laser spot over the sample edge and analyzing the intensity profile of the transmitted light, an optical resolution of 30 μm, close to the theoretical resolution limit of 20 μm of the optical system with a numerical aperture of NA = 0.025, has been determined. 
%The Faraday rotation was measured using a differential detector system consisting of a $\lambda$/2 plate, a focusing lens, a polarizing beamsplitter, and two photodiodes. 
%The probing spot was scanned in two lateral directions with a step size of 20 microns across a 1.4mm by 1.4mm rectangular area covering the disc.
%
%The angle of the Faraday rotation is proportional to the magnetization component parallel to the beam direction of propagation. 
%For the used experimental geometry this means that the setup is sensitive to the out-of-plane component of the magnetization, $M_{\perp}$. 
%In order to calibrate the detected Faraday rotation signal in terms of the out-of-plane magnetization angle, we have tilted the magnet by 15° out of the sample plane and recorded the remagnetization curve, measured by sweeping the applied magnetic field from -0.3 to 0.3 T and back. Using a single-domain model with literature values for the anisotropy constant and saturation magnetization to calculate the remagnitization curve, we are able to calibrate the measured Farady signal in millivolts measured at the detector to the angle of out-of-plane precession of the magnetization.
%
%Lastly, for the space-resolved measurements, the phase of the microwave signal has been adjusted to maximize the detected SW amplitude at the center of the sample. 
%For the field dependent spectra, we maximize the SW amplitude for the field value corresponding to the peak of the main mode.

\appendix
\section{Appendix A: Magneto-optics}
An isotropic medium becomes optically anisotropic when placed in an external magnetic field.
%
Maxwell's macroscopic equations are
\begin{align}
\nabla \cdot \vec{D} &= \rho_{f} \\
\nabla \times \vec{E} + \frac{\partial \vec{B}}{\partial t} &= 0 \\
\nabla \cdot \vec{B} &= 0 \\
\nabla \times \vec{H} - \frac{\partial \vec{D}}{\partial t} &= \vec{J}_{f}.
\end{align}
with the constitutive relations
\begin{align}
\vec{D} &= \epsilon_{0} \vec{E} + \vec{P} \\
\vec{H} &= \frac{1}{\mu_{0}} \vec{B} - \vec{M}.
\end{align}
%
Jones matrices for a polarizer, sample, retarder, and detector:
\begin{align}
E_{(O)} &= \left[ \begin{matrix} e^{i \frac{\delta}{2}}&0\\ 0&e^{-i \frac{\delta}{2}} \end{matrix} \right] \left[ \begin{matrix} r_{ss}&r_{sp}\\ r_{ps}&r_{pp} \end{matrix} \right] \left[ \begin{matrix} \cos{\alpha}\\ \sin{\alpha} \end{matrix} \right] \\
&= \left[ \begin{matrix}(r_{ss} \cos{\alpha} + r_{sp} \sin{\alpha}) e^{i \frac{\delta}{2}} \\(r_{ps} \cos{\alpha} + r_{pp} \sin{\alpha}) e^{-i \frac{\delta}{2}} \end{matrix} \right].
\end{align}
If $\alpha = 0$ then the light is s-polarized and we find
\begin{equation}
E_{(O)} = \left[ \begin{matrix} r_{ss} e^{i \frac{\delta}{2}}\\ r_{ps} e^{-i \frac{\delta}{2}} \end{matrix} \right].
\end{equation}

\begin{thebibliography}{99}

% Landau Lifshitz, E&M
\bibitem{LL_E&M}
E. M. Lifshitz, L. P. Pitaevskii, and L. D. Landau,
{\it Electrodynamics of Continuous Media}
(Elsevier, New York, 1985).

% SKEM Tamaru
\bibitem{SKEM_Tamaru}
S. Tamaru, J. A. Bain, R. J. M. van de Veerdonk, T. M. Crawford, M. Covington, and M. H. Kryder, 
% title
{\it J. Appl. Phys.}, {\bf 91}, 8034 (2002).

% SKEM C. H. Back group
\bibitem{SKEM_Back}
I. Neudecker, K. Perzlmaier, F. Hoffmann, G. Woltersdorf, M. Buess, D. Weiss, and C. H. Back, 
% title
{\it Phys. Rev. B}, {\bf 73}, 134426 (2006).


\end{thebibliography}

\end{document}

%% Kalinikos theory, part 1
%\bibitem{K_1980} 
%B. A. Kalinikos,
%Excitation of propagating spin waves in ferromagnetic films.
%{\it IEE Proc. H} {\bf 127}, 4 (1980).
%
%% Kalinikos theory, part 2
%\bibitem{KS_1986}
%B. A. Kalinikos and A. N. Slavin,
%Theory of dipole-exchange spin-wave spectrum for ferromagnetic films with mixed exchange boundary conditions. 
%{\it J. Phys. C} {\bf 19}, 7013 (1986).

% Figure and Figure Reference example
%The resultant plot appears in Fig. \ref{fwhm_ampl} and as a log-log graph in Fig. \ref{fwhm_ampl_log}.
%\begin{figure}
%  \centering
%  \includegraphics[angle=-90,width=100mm]{fwhm_ampl_log.eps}
%  \caption{Peak amplitude of $\| \chi_{xx} \|^2$ as a function of linewidth, log-log plot. \label{fwhm_ampl_log}}
%\end{figure}

%Summary. A Ti:Sapphire mode-locked laser (Coherent Mira 900) produces quasimonochromatic light pulses with a wavelength of 800 nm and duration of 210 fs. The pulses are frequency doubled and focused onto the sample by an objective with numerical aperture (NA = ?), thereby achieving a diffraction limited spot-size $\sim$ 200nm. The reflected light is collected by the same objective, (QWP?), then split by a Wollaston prism onto two intensity photodiodes.

%"Voigt effect (linear magnetic birefringence and dichroism quadratic in magnetic field)..."

%"The classical theory of magneto-optical effects is based on the electromagnetic theory of the interaction of light with a medium. In this theory, light is described as an electromagnetic wave and the optical properties of the medium are expressed in terms of the permittivity or conductivity tensors."

%\noindent Section 0: Kerr/Faraday effect
%-Microscopic origins
%-Classical/QM treatment
%-Gyrotopy of materials, polarization spectroscopy
%-Dichroism and birefringence...other applications of ``polarization spectroscopy"
%\noindent Section 1: Schematic
%-FMR Kerr spectroscopy
%-Conceptual schematic
%-Real schematic
%\noindent Section 2: Laser
%-Ti/Sapphire laser cavity
%-Beam focusing
%-Spatial resolution limit
%\noindent Section 3: Polarization analysis
%-Jones matrix calculation
%-Stokes parameters
%-Detector
%\noindent Section 4: Excitation
%-Temporal resolution limit
%\noindent Section 5: Synchronization
%-Synchrolock box/Phase-locked loop (PLL)
%-Supporting electronics
%\noindent Section 6: Signal Analysis
%-Measurement geometries
%-Lock-in technique
%\noindent Section 7: Sample control
%-Stage control, stage resolution
%\noindent Section 8: LabView programs
%-Program documentation